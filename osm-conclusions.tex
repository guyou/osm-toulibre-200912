%% osm-conclusions.tex
%%


\section{Conclusions}

\frame{ \heading{Perspectives} \vfill

  \begin{itemize}
  \item processus de revue, alerte email en cas de modif zone reviewé
  \item services de «données néttoyées» pour utilisations critiques
    \begin{itemize}
    \item validation de la qualité des ajouts/suppressions de données
    \item homogénisation des tags
    \end{itemize}
    
  \item meilleure gestion des rollback en cas d'erreur volontaire ou malveillant

  \item[{\color{darkyellow}\Lightning}] outils pour gérer un «édit war»:
    \begin{itemize}
    \item prévention: permettre des rendus par langue
    \item locks par objet/zone géographique contestée 
    \item désignation de modérateurs
    \end{itemize}
  \item éventuel changement de licence: CCBYSA $\rightarrow$ ``Open Database Licence''
  \item création d'une association française afin de disposer d'un statut juridique pour dialoguer avec des ayants-droit
  \end{itemize}
}



\frame{ \heading{Conclusions} \vfill

  Intérêts des cartes libres:
  \begin{itemize}
  \item utilisation libre des données cartographiques
    \begin{itemize}
    \item créer un logiciel de navigation libre
    \item créer des cartes spécifiques liés à ses propres intérêts
    \item illustrer des documents libres
    \end{itemize}

  \item permettre de corriger des erreurs dans les cartes (rues
    devenues en sens unique, \ldots{})
  \end{itemize}

  \bigskip
  
  Limites:
  \begin{itemize}
  \item couverture encore très inférieure aux solutions propriétaires en France
  \item qualité des données non garantie (vandalisme, \ldots{})
  \end{itemize}

}


% Convention d'Aarhus
%   Signée le 15 juin 1998 ratifiée en France par la
%   loi du 28 février 2002 et du 26 octobre 2005
% • les trois piliers de la Convention
%    – développer l’accès du public à l’information détenue
%      par les autorités publiques,
%    – favoriser la participation du public à la prise des
%      décisions liées à l’environnement,
%    – étendre les conditions d’accès à la justice.



\frame{ \heading{Pour en savoir plus} \vfill

  \begin{itemize}
  \item Site web: \url{http://openstreetmap.org/}
  \item Courriel: \url{talk-fr@openstreetmap.org}
  \item IRC: \#osm-fr sur le serveur \texttt{irc.oftc.net}
  \item Animations de l'évolution du projet: \url{http://www.jabberworld.org/osm/}
  \end{itemize}
}


\frame{ \heading{Remerciement} \vfill


  \begin{beamerboxesrounded}[width=\textwidth,scheme=alert,shadow=true]{} \sffamily\tiny
  Cette présentation est diffusable selon les termes de la license
  CC-BY-SA 2.0. Des éléments ont été repris de présentations préparées
  par:
  \begin{itemize}
  \item Émmanuel Garette
  \item Frederik Ramm
  \item Jochen Topf
  \item Sylvain Beorchia \& Thomas Walraet
  \end{itemize}

  \end{beamerboxesrounded}

}


%% EOF
